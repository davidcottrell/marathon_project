\documentclass[12pt,titlepage]{article}

% \usepackages21{fancyhdr}

\usepackage[printwatermark]{xwatermark}

\usepackage{grffile}
\usepackage{xcolor}
\usepackage{lipsum}
\usepackage{times}
\usepackage{soul}
\usepackage{epsfig}
\usepackage{rotating}
\usepackage{url}
\usepackage{latexsym}
\usepackage{graphicx}
\usepackage{amsfonts}
\usepackage{amsmath, amsthm, amssymb}
\usepackage{fullpage}
\usepackage{setspace}
\usepackage{natbib}
\usepackage{longtable}
\usepackage{keyval}
\usepackage{caption,subcaption}
\usepackage{arydshln}
%%\usepackage[hyphenbreaks]{breakurl}

%% allow urls to get broken on hyphens
\usepackage{hyperref}
\def\UrlBreaks{\do\/\do-}


%% APSR submission: no commas in citations between name and year
%% See http://merkel.zoneo.net/Latex/natbib.php
\bibpunct{(}{)}{;}{author-year}{}{;}

% the opening bracket symbol, default = (
% the closing bracket symbol, default = )
% the punctuation between multiple citations, default = ;
% the letter `n' for numerical style, or `s' for numerical superscript style, any other letter for author-year, default = author-year;
% the punctuation that comes between the author names and the year
% the punctuation that comes between years or numbers when common author lists are suppressed (default = ,);

\usepackage{footmisc}
\renewcommand{\footnotelayout}{\doublespacing} % set spacing in footnotes
\newlength{\myfootnotesep}
\setlength{\myfootnotesep}{\baselineskip}
\addtolength{\myfootnotesep}{-\footnotesep}
\setlength{\footnotesep}{\myfootnotesep} % set spacing between footnotes

% make footnote font size same as regular font size in text
\renewcommand{\footnotesize}{\normalsize} 


%% Use this for a "DRAFT" watermark
%% \newwatermark[allpages,color=pink!50,angle=45,scale=5,xpos=-25,ypos=40]{DRAFT}

%% List all locations for graphics here
\graphicspath{ {../plots/} }


\begin{document}
\sloppy
\thispagestyle{empty}

%% APSR submission requires double-spaced footnotes
%%\newcommand{\footnote}[1]{\footnote{\doublespacing #1}} %% <-- note \doublespacing here.

\renewcommand{\topfraction}{.85}
\renewcommand{\bottomfraction}{.7}
\renewcommand{\textfraction}{.15}
\renewcommand{\floatpagefraction}{.66}
\renewcommand{\dbltopfraction}{.66}
\renewcommand{\dblfloatpagefraction}{.66}

\newcommand{\yi}{\ensuremath{Y_i}}

% \urldef\myurlncsl1\url{foo%.com}
% \begin{document}
% text\footnote{WWW: \myurl}


\title{\Large{All in the family:\\Anna and Lisa Hahner's finishing
    times in\\the 2016 women's Olympic
  marathon}}\author{David Cottrell\thanks{Postdoctoral Research
  Fellow, Program in Quantitative Social Science, Dartmouth College,
    6108 Silsby Hall, Hanover, NH
    03755 (\texttt{david.cottrell@dartmouth.edu}).} \and Michael C.\
  Herron\thanks{Visiting Scholar, Hertie School of Governance, Berlin,
    Germany, and Professor of Government, Dartmouth College, 6108
    Silsby Hall, Hanover, NH 03755
    (\texttt{michael.c.herron@dartmouth.edu}).}}


\maketitle \doublespacing 

%\begin{abstract} 
% \noindent
%The abstract
%\end{abstract}

%\newpage


\begin{quote}
  \emph{``I invested all I had and 300 meters before the finish line,
    I was next to Lisa. It was a magical moment that we could finish
    this marathon together. We did not think about what we were
    doing.'' -- Anna Hahner}
\end{quote}


\section*{Introduction}

On August 14, 2016, at 9:30 in the morning, the Women's Olympic
marathon kicked off in Rio de Janeiro, Brazil, when 156 runners from
80 countries across the world left the starting line en route to their
destination 42.195 kilometers away. Two hours, twenty-four minutes,
and four seconds later, Jemima Sumgong of Kenya would be the first to
cross the finishline and take home gold; Sumgong was just 3.5 minutes
behind her prior personal best time in the marathon. Approximately 21
minutes later, twin marathoners from Germany, Anna and Lisa Hahner,
would cross the finishline together, holding hands and celebrating a
personal victory. Although the Hahners would finish 81st and 82nd,
respectively, with times slightly more than 18 minutes slower than
corresponding personal bests, Anna Hahner would describe their joint
finish as a ``magical moment.''

The media quickly picked up on the Hanher story as an image of the
beaming twins finishing hand-in-hand captured a public audience. Many
believed the moment was a reflection of the Olympic spirit.

%% Honigkuchenpferde reference
%%
%% https://www.welt.de/sport/olympia/article157669264/Das-falsche-Laecheln-der-deutschen-Lauf-Zwillinge.html 

Not everyone agreed with this rosy interpretation. The twins' happy
facial expressions at the finish were portrayed as a bit
contrived---smiling like ``Honigkuchenpferde,'' cookies in the shape
of a horse---and the sports director of the German Athletics
Federation, Thomas Kurschilgen, stirred up controversy when he
suggested that the Hahners' photo-finish was no coincidence.
Kurschilgen averred that the twins slowed down so as to finish
simultaneously and create a spectacle which would ``generate media
attention.'' Kurschilgen justified his charge with the fact that the
twins ran in the Rio marathon at least 18 minutes slower their
personal best times prior to the
Olympics.\footnote{\url{https://www.nytimes.com/2016/08/17/sports/olympics/twins-finish-marathon-hand-in-hand-but-their-country-says-they-crossed-a-line.html}}
Not surprisingly, Kurschilgen's accusations were denied by the Hahner
twins, who claimed that their simultaneous finish was simply an
unintended coincidence.

%% Honigkuchenpferd quote from here:
%%
%% https://www.welt.de/sport/olympia/article157669264/Das-falsche-Laecheln-der-deutschen-Lauf-Zwillinge.html

What happened in the women's Olympic marathon, and how might we
develop a statistical approach that assesses whether the Hahner twin's
finish was coincidental or intentional?  These two interpretations are
clearly at odds. If the former, then the Hahners are to be celebrated
and their finish treated as an expression of the spirit behind the
Olympic games. If the latter, though, then the twins may have violated
this spirt by not trying hard enough. It is perhaps too easy for us to
write such a glib sentence---neither of us can fathom being able to
complete a marathon anywhere in the vicinity of two and a half
hours---but we nonetheless want to know what the data from the Olympic
marathon tell us.  Was the Hahner finish in the 2016 women's
Olympic marathon a lovely coincidence or something else?

Among female Olympic marathoners, the Hahner twins were not alone in
their familial ties, and we touch on this in the analysis that
follows.  The marathon also featured twins from North Korea, Kim
Hye-song and Kim Hye-gyong, who posted identical times and finished
10th and 11th in the race, respectively. The Kim finish, unlike the
Hahner finish, appears devoid of post-race controversy. Moreover,
three triplets from Estonia competed in the Rio marathon, although
only two, Lily Luik and Leila Luik, finished it, in 97th and 114th
place, respectively. The third Estonia triplet, Liina Luik, recorded
what is known as a DNF---an abbreviation that means did not finish, a
term that we will use throughout this article.

\section*{Marathon data and our research design}

For each participant who started the women's Olympic marathon, we know
several things: personal best marathon time prior to the 2016 Olympic
games; split times from the Rio marathon course at 5 kilometers, 10
kilometers, and so forth; and, finishing time. We cannot directly
observe the effort that an individual put into the race, and we do not
know why some runners have DNF results.  Some runners may have injured
themselves on the course and accordingly dropped out, and others may
have dropped out, uninjured, in anticipation of an unsatisfactory
result. Of the 156 marathon starters, 133 completed the race and 23
DNFed at various locations throughout the course. The overall DNF rate
was thus $\frac{23}{156} \approx 0.15$, and the relatively small
sample size at our disposal means that a 95\% confidence for this rate
is fairly wide, namely, $\left(0.098, 0.22\right)$. 

The Kurschilgen accusation against the Hahner twins has two
components, that the twins finished simultaneously, holding hands and
smiling, \emph{and} that they ran slowly.  We suspect that Kurschilgen
would not have expressed ire at the Hahners had they finished in 1st
and 2nd place in Rio, hand-in-hand with wide grins, but of course we
do not know this.  As such, our investigation of the charges that
Kurschilgen offered will distinguish between the idea of slow finish
versus a simultaneous finish.

Our research design has two components.  In the first, we present
visualizations that describe various features of the Hahner twin's
results, and an important element of our visualizations is the
difference between a runner's Rio time and her prior personal best
time in the marathon.  Overall, our visualizations suggest blah blah
bhah.  We then turn to modeling and in particular a regression-based
simulation.  In simulated Olympic marathons, which we develop based on
relationships (excluding the Hahner and Kim twins and Luik triplets)
between known personal best times and observed Rio finishing times, we
find that the Hahner twins finished suspiciously close to each other
given the disparty between their personal best times.  OR THEY DID
NOT.  Finally, we return to Kurschilgen claims  about the Hahners and
offer our thoughts about their validity.

\section*{Visualizing the Olympic marathon}

Figure \ref{fig:scatter} contains two plots, both of which describe
how the Rio Olympic women's marathon results varied as a function of
athletes' personal best times and their half marathon split times.
The points in each plot are colored by twin/triplet status, and both
plots contain second-order polynomial regression lines.  We treat an
athlete's prior personal best marathon time as a measure of the
athlete's underlying marathon talent, and we interpret an athlete's
half marathon split as a measure of the athlete's skill at marathon
running on August 14, 2016, the date of the women's marathon in Rio.

\begin{figure}[!ht]
  \caption{Olympic results and measures of marathon ability}
  \label{fig:scatter}
  \begin{subfigure}{.5\textwidth}
    \centering
    \includegraphics[width=\textwidth, keepaspectratio]{scatter_plot.pdf}
    \caption{Finishing and personal best times}
    \label{fig:45degreeplot}
  \end{subfigure}
  \begin{subfigure}{.5\textwidth}
    \centering
    \includegraphics[width=\textwidth, keepaspectratio]{scatter_plot_half.pdf}
    \caption{Finishing and half marathon split times}
    \label{fig:45degreeplot_half}
  \end{subfigure}
\end{figure}

Considering first Figure \ref{fig:45degreeplot_half}, note that this
figure includes a solid 45-degree line as the horizontal and vertical
axes in the figure correspond to full marathons.  The rug marks on the
bottom of the figure denote the personal best times of runners who
DNFed.  Given the paucity of points (five of them) below the 45-degree
line, the vast majority of Olympic marathoners ran slower in Rio
compared to their personal bests.  Relative to personal best times,
the Hahner twins were definitely on the slow side, but a number of
runners had greater differences between their Olympic times and their
personal bests than the Hahners.  These runners are denoted with
squares in Figure \ref{fig:45degreeplot}, and there are 13 such
symbols, highlighting approximately 10\% of the racers who completed
the marathon.  The figure shows moreover that there were two women who
had personal best times slightly faster than the Hahner's and yet
finished after both of the German women.  Although Figure
\ref{fig:45degreeplot} suggests that the Hahner twins were slower than
one would have expected given their previous best marathon times, it
is not consistent with the accusation that they dramatically slowed
down in the Rio marathon.

% > df <- read.csv("data/times.csv", stringsAsFactors = FALSE) %>% tbl_df()
% > df <- df[df$BIB != 1172,]
% > sort(dta$FINAL - dta$PB) / 60
%   [1] -2.3333333 -2.2166667 -2.0666667 -1.1166667 -0.4000000  0.2166667
%   [7]  0.6333333  0.9000000  1.0166667  1.3666667  1.5166667  1.7166667
%  [13]  1.7666667  1.8500000  2.1166667  2.4666667  2.5333333  3.0166667
%  [19]  3.0500000  3.1666667  3.2666667  3.2833333  3.3666667  3.4500000
%  [25]  3.6833333  3.9000000  3.9833333  4.2000000  4.2666667  4.3833333
%  [31]  4.4666667  4.5833333  4.6166667  4.6333333  4.6833333  4.7333333
%  [37]  4.7333333  4.8333333  5.0666667  5.0833333  5.0833333  5.1000000
%  [43]  5.1333333  5.3500000  5.3500000  5.4000000  5.7166667  5.7333333
%  [49]  5.9333333  6.0000000  6.0333333  6.1166667  6.3333333  6.3500000
%  [55]  6.5000000  6.5333333  6.6000000  6.6666667  6.7166667  6.7833333
%  [61]  6.8166667  6.9500000  7.1000000  7.1500000  7.2000000  7.3500000
%  [67]  7.4500000  7.5666667  7.6000000  7.6000000  7.6333333  7.7500000
%  [73]  7.8833333  7.9666667  7.9833333  8.3000000  8.6000000  9.3000000
%  [79]  9.5500000  9.5833333  9.6000000  9.7166667  9.9500000  9.9833333
%  [85] 10.3166667 10.5333333 10.6666667 10.7000000 10.8333333 10.9666667
%  [91] 11.1500000 11.1833333 11.4666667 11.7166667 12.1666667 12.6000000
%  [97] 12.9166667 13.2166667 13.2833333 13.3833333 13.7666667 13.8666667
% [103] 13.9166667 14.0500000 14.0833333 14.5666667 14.6833333 14.7500000
% [109] 15.0000000 15.0500000 15.5000000 15.9000000 16.2000000 16.9000000
% [115] 17.4500000 18.1000000 18.6666667 18.7166667 18.8000000 18.9333333
% [121] 20.1500000 20.6333333 20.7000000 20.8500000 21.3500000 23.4166667
% [127] 24.9333333 25.6500000 26.5666667 26.7666667 29.2166667 30.7000000
% [133] 30.9666667
% > 

One limitation of using personal best times as indicators of
underlying athletic talent is that these times are potentially
confounded by the marathon courses at which they were set (some*sh
courses, like the Berlin marathon, are known for fast times) and race
conditions like weather.  In addition, personal best times may not
capture raceday idiosyncrasies that might affect individual runners.
With that in mind, Figure \ref{fig:45degreeplot_half} plots marathon
finishing times against half marathon splots.  This figure has a
regression line as before but no 45-degree line.

Figure \ref{fig:45degreeplot_half} shows that there was nothing
abnormal about the Hahner twins' overall finishing times, conditional
on their half marathon split times.  This is consistent with the
previous Figure \ref{fig:45degreeplot} and inconsistent with the
accusations against the Hahner twins, at least the part of the
accusation that focuses on their excessively slowing down.

If the Hahner twins did not slow down excessively, might they have
run somewhat strategically at the end of the race in order to generate
a simultaneous finish?  We now offer a visualization that speaks to
this question.

The personal best times of the Hahner twins were 115 seconds apart and
their official finishing times were separated by one second.  Is such
a 115 to one compression typical among pairs of runners or was it
unusual?  We might ask, are there other pairs of runners who had a
difference between personal best times of 115 seconds apart and, if
so, how close where their finishing times?

Of the 133 marathon finishers, there are $\binom{133}{2} = 8,778$
pairs of runners.  Of these and ignoring the Hahner twins, ten had
exactly a 115 second gap in personal best times.  Of these ten pairs,
differences in finishing times in seconds are as follows: 36, 93, 172,
319, 379, 459, 552, 671, 675, and 739.  In other words, of all pairs
of runners in the Rio marathon who had a personal best difference that
was equivalent to the Hahner twins' difference, the twins had the
greatest compression based on finishing time.

% df2b[abs(df2b$pb_diff)==115,]
% # A tibble: 11 × 8
%                       name_i              name_j pb_diff result_diff
%                        <chr>               <chr>   <int>       <int>
% 1                Anna Hahner         Lisa Hahner    -115          -1
% 2                  Lily Luik Mariya Korobitskaya     115          36
% 3        Lisa Jane Weightman      Lanni Marchant    -115          93
% 4           Irina Smolnikova      Viviana Chávez     115        -172
% 5              Sonia Samuels    Helalia Johannes     115        -319
% 6  Jessica Draskau-Petersson       Gladys Tejeda     115         379
% 7        Maryna Damantsevich       Gladys Tejeda     115         459
% 8          Panayióta Vlaháki     Matea Matosevic    -115         552
% 9         Nataliya Lehonkova       Olha Kotovska     115         671
% 10               Anja Scherl      Desireé Linden     115         675
% 11              Rose Chelimo    Helalia Johannes    -115        -739
% # ... with 4 more variables: diff_in_diff <int>, consecutive <lgl>,
% #   twins <chr>, PERCENTILE <dbl>
% > 

\begin{figure}[!ht]
 \caption{Differences in personal best times and finishing times}
 \label{fig:diffdiffscatter}
 \centering
 \begin{subfigure}{.5\textwidth}
 \includegraphics[width=\textwidth, keepaspectratio]{diff_in_diff_scatter_plot.pdf}
 \end{subfigure}
\end{figure}

We can generalize this result by looking at all pairs of runners in
the marathon.  For the 8,778 pairs of 133 finishers, Figure
\ref{fig:diffdiffscatter} plots differences in finishing times against
differences in personal best times, and pairs of twins/triplets are
identified by the same color scheme we used earlier.

Consider first the Hahner twins.  They are effectively located on the
horizontal axis because their difference in finishing times is one
second.  However, there are many points about the Hahner's orange dot,
and this shows that, conditional on a 115 second difference in
personal best times, most marathoners did not have close finishing
times like the Hahnes.  Some pairs of runners with 115 second personal
best differences had finishing time differences of 1000 seconds, i.e.,
in excess of 15 minutes.  Of course the points in Figure
\ref{fig:diffdiffscatter} are not independent, but they provide a
sense of the dispersion in finishing time differences that one can
expect conditional on differences in personal best times.

Thinking about the accusations leveled against the Hahner twins,
Figure \ref{fig:diffdiffscatter} suggests that Anna and Lisa Hahner
did indeed run with an eye on each other.  In fact, the same can be
said of the Kim twins, who ran seemingly in lockstep throughout the
entire Rio marathon.  The North Korean twins had a personal best
difference of 53 seconds and a finishing time difference of literally
zero seconds.  Beyond these twins, there were eight pairs of runners
with a 53 second personal best difference, and resulting finishing
time differences are as follows: 9, 51, 228, 340, 352, 571, 662, and
751.  As in the Hahner case, the Kim twins compressed their finishing
times more than any other pair of runners with a similar personal best
difference.

To get a sense of the lockstep nature of the Kim twins race, Figure
\ref{fig:secondsbehind} describes each runner's status at the various
split times on the marathon course.  Each dot in the figure depicts a
recorded split and the number of seconds each runner was behind the
race leader at the time.  There are more dots at earlier splits due to
the accumulation of DNFs.  The usual color scheme applies here, and
the Estonian DNF prior to the half marathon split is evident.  The
orange dots in the figure represent the Hahner twins, and as noted in
the figure Lisa Hahner was ahead of her sister through 15 kilometers
at which point Anna surged ahead.  Figure \ref{fig:secondsbehind}
contains two red dots representing the North Korean Kim twins, but
this is not visually apparent because the Kim twins had identical
split times during the entire race marathon.

\begin{figure}[!ht]
  \centering
  \caption{Runner status by split}
  \label{fig:secondsbehind}
  \includegraphics[scale = 1]{seconds-behind.pdf}
  \begin{flushleft}
    \emph{Note: each dot represents one runner at a split.  DNFs are
      not pictured, and splits are not to scale.}
  \end{flushleft}
\end{figure}


\section*{The probability of an unintentional simultaneous finish}

Thomas Kurschilgen's accusations of an intentional finish stems from
two observations. First, the Hahner twins finished slower than he
expected given respective personal bests. Second, the twins finished
together at the exactly same time. Kurschilgen clearly believed that
neither of these events would have occurred had both Hahner run the
marathon independently, absent coordination. By his logic, the German
twins should have run faster and not have finished simultaneously.

On the other hand, only a handful of runners completed the Rio
marathon with times that were faster than their recorded best times.
Thus, 18 minutes behind a personal record may not be the outlier that
Kurschilgen claimed it to be. Moreover, there is good reason to
suspect that an unintended simultaneous finish was more likely for the
Hahner twin than it would be for any other set of runners. After all,
these two women are twins with presumably similar abilities. Not only
do they train together, but their pre-Rio best times are less than two
minutes apart. While Anna may be slightly faster than Lisa measured by
personal bests, we might expect the difference in their Rio result to
be just as close as the differences in their recorded bests. And given
random variation in finishing times, a simultaneous finish might not
be out of the ordinary.

Even though they seem similar, finishing at very similar times and
finishing together are different phenomena. If, for example, both
Hahners were of similar ability to each other and also to many other
runners, then we might expect similar finishing times yet not
necessarily similar placements. The latter will be a function of the
extent to which all runners on the Rio course have similar talent
levels.  This point is an important one and will be evident in the
results that follow.

To test the assertion that the Hahner twins paced themselves to finish
at the same time and with back-to-back placements, we need to compute
the probability that such a result would have occurred
unintentionally. Hence we need to know the probability distribution of
Anna and Lisa's finishing times if their runs had been independent of
each other. The challenge, of course, is that this distribution is
unknown.  

\section*{Modeling}


The Hahner twins either acted independently $I$ or coordinated their
times $C$; these are the two possible states of the race
$\psi = \{I, C\}$. Given that we observed Anna and Lisa's finishing
times, $Y_A = y_A$ and $Y_L = y_L$, we want to know the probability
that they ran independently, as they say they did. Hence, we are
looking to determine, 

$$P(\psi = I \mid y_A  \cap y_L )$$

However, to determine this, we need to have some understanding of a
likelihood function that specifies Rio finishing times. We want to
know the likelihood of Anna's and Lisa's final times given
independence $P(y_A \cap y_L \mid \psi = I )$. We can estimate this
function with a few assumptions. First, we assume that under
independence, any given runner's final time $Y_i$ is conditional on
his/her running ability plus noise. Specifically, we assume that the
$Y_i$ is a linear function of the runner's ability $X_i$ plus a
normally distributed error term $e_i \sim N(0, \sigma)$. Second, we
assume that every runner shares the same linear relationship - meaning
the slope and intercept remain constant across runners. Third, we
assume that the error term is drawn from a common distribution across
runners. Hence, luck and misfortune are drawn from the same
distribution. Therefore,

$$Y_i \sim N(\beta_0 + X_{i}\beta_1 + e_i, \sigma)$$

We also assume that a runner's ability $X_i$ can be measured precisely
by her best marathon performance leading up to the Olympics. 

%% Any measurement error must therefore be negligible.

If Anna and Lisa intentionally slowed down as a result of coordination
then we would likely observe $y_A > E(Y_A \mid \psi = I) $ and
$y_L > E(Y_L\mid \psi = I )$. In other words, the final times that
Anna and Lisa recorded in the race would be greater than we would
expect if they had run independently.

Moreover, if Anna and Lisa coordinated to finish simultaneously, then
the difference between the two sisters' final times would be less that
the expected difference had they run independently. Therefore, under a
coordinated finish we would expect $\left|y_A - y_L\right| < E(\left|Y_A - Y_L\right| \mid \psi = I )$.

\section*{Did Anna and Lisa intentionally slow down?}

According to Kurschilgen, Anna and Lisa underperformed in the Rio marathon. He claimed that because their goal was to finish simultaneously rather than finish at their fastest pace, their times were slower than they otherwise would have been.  He claimed that the twins were simply trading speed for a photo-finish.

If this were the case, the function generating Anna and Lisa's final times would deviate from the function that generated everyone else's final times.  Given everyone else would draw their times from the independent distribution  $Y_i \sim N(\beta_0 + X_{i}\beta_1 + e, \sigma)$,  Anna and Lisa would draw from a distribution of times that are slower in expectation. Hence, they would lie well-above the line that links a runner's performance in Rio to their previous best performance.    

\begin{figure}[!ht]
  \caption{Relationship between Personal Best and Result  -- caption?}
  \label{fig:scatter}
  \begin{subfigure}{.5\textwidth}
    \centering
    \includegraphics[width=\textwidth, keepaspectratio]{scatter_plot.pdf}
    \caption{Rio finishing times and personal best times}
    \label{fig:45degreeplot}
  \end{subfigure}
  \begin{subfigure}{.5\textwidth}
    \centering
    \includegraphics[width=.975\textwidth, keepaspectratio]{studentized_residuals.pdf}
    \caption{Studentized residuals}
    \label{fig:studentizedresiduals}
  \end{subfigure}
\end{figure}

Figure \ref{fig:45degreeplot} displays the relationship between a
runner's personal best (horizontal axis) and her final result
(vertical axis).  The 133 finishers are plotted as grey points, and
the 23 DNFs are depicted below along the horizontal axis with
tick-marks.  The dashed black line with grey confidence intervals
represent ordinarily least squares estimates of the linear
relationship between racer's finishing and personal best times.

One can see in Figure \ref{fig:scatter} that there is a clear
relationship between a runner's personal best time and her performance
in the Olympics.  While there is a significant amount of noise in the
relationship, the Hahner twins' dots are well above the regression's
fitted line notwithstanding this point.  They finished the Rio
marathon much slower than they should have, per their pre-Olympic
personal best times.  In fact, if we sort studentized residual, we can
see in Figure \ref{fig:studentizedresiduals} that the Hahner twins are
in the tail-end of the residual distribution.  Anna Hahner's residual
deviation is greater than 91.7\% of the runners who completed the
marathon while Lisa Hahner's deviation is greater than 86.5\%.  Hence,
the twins appear to have finished at a much slower pace than expected,
which is what we might expect if they were coordinating their runs.
The North Korean Kim twins are also in the tail of the residual
distribution---but the left tail.  These twins finished much faster
than expected. 

\section*{Did Anna and Lisa intentionally finish together?}

Anna and Lisa finished the marathon at an unusually slow rate, which is what we might expect if the twins intentionally slowed their pace to finish simultaneously.  However, the observation of a slow finish does not necessarily imply that it was an intentional result .  It only suggests that Kurschilgen's claim regarding this issue holds up in the data.

Perhaps it would be easier to refute the claim if we could show that the Twins's back-to-back finish was not such an unusual event.  Given that Anna and Lisa were so close in their personal best times, it may seem reasonable to expect that they would also be close in their finishing times.   Hence, one way to determine if the twins's finish was intentional is to quantify the likelihood that such a close finish would occur by chance given the differences in their ability. 

We can attempt to quantify this likelihood by looking at how differences in the personal best time of two runners translate into differences in Olympic times .  If Anna and Lisa's finish was unintentional we'd expect that such a close finish would be unlikely given their difference in personal best times.   Yet, how does the difference in personal best times relate to the difference in finishing times?

One way we can estimate this relationship is to look at all of the differences that we observe in the marathon.  We can take the difference in personal best time for every combination of runners in the data and look at the difference in their final times.  We should expect that compared to the difference in their personal best times, the difference in Anna and Lisa's final times should be commensurate with the conditional difference of all other runners. %This needs a better explanation.

Therefore, we find all combinations of runners in the marathon and calculate two quantities: 1) the difference in their personal best time and 2) the difference in their final time. We plot the relationship between the two quantities in Figure \ref{fig:diffdiffscatter}.   All 8,778 dyadic combinations of runners are displayed by grey points in the scatter plot.   The solid black line indicates a one-to-one relationship, where every point would fall if the difference in final times were systematically equivalent to the difference in personal bests.  


% \begin{figure}[!ht]
%  \caption{Relationship between Difference in Personal Best and Difference in Result}
%  \label{fig:diffdiffscatter}
%  \centering
%  \begin{subfigure}{.5\textwidth}
%  \includegraphics[width=\textwidth, keepaspectratio]{diff_in_diff_scatter_plot.pdf}
%  \end{subfigure}
% \end{figure}

We can see from the yellow dot that the Hahner Twins are just below this one-to-one line.  Hence, the twins' final times were closer to each other than their personal best times; but not by much.  In fact, this might be an expected difference if we expected there to be a one-to-one relationship.     However, although there is a clear relationship between the difference in final time and the difference in personal best, it is not a strong one-to-one relationship.  We've added a dotted smoother to display a cubic splines estimate of the conditional averages.  While the Hahner twins may be close to the one-to-one line, they are much further away from the average set of runners.  Hence, we would expect for any pair of runners, the difference in their final results should be much greater than their difference in their personal best.  However, the Hahner twins do not fit this expectation.   Yet, how different from this expectation are they?

One way we can quantify the degree to which the twins are outliers is by comparing their difference in difference to every other pair of runner's difference-in-differences.  We calculate the difference-in-differences for each pair of runners by simply subtracting their difference in result (y-axis) from their difference in personal best (x-axis).  Therefore positive difference-in-differences mean that the runners ended up further apart than their difference in personal bests.  Negative difference-in-differences mean that the runners end up closer together.  Hence, the Hahner twins have a difference-in-difference of -114 seconds.  They finished 114 seconds closer together than the distance between their personal bests.  

We arrange all 8,778 difference-in-differences from small to large in the first plot in Figure \ref{fig:diffdiff}. The Hahner twins are denoted by the yellow point.  Using this plot we can visualize where the twins fall within the distribution of difference-in-differences.  It is clear that they are closer to the lower tail of the distribution.  However, compared to all other difference-in-differences, their result is not that extreme.  Nearly 21\% of the difference-in-dfferences were less than the that of the twins'. Hence, we'd expect at least one in five pairs of runners to have a reduction in difference at least as great as the twins.

However, if we limit the observations to only those runners who had personal best times that were at least as close to each other as the Hahner twins, we get a different perspective.  Among those runners,  the Hahner twins reduced the distance between them to the greatest extent.  There was no other pair of runners whose reduced the difference in the final time relative to their difference in personal bests more than them.         

\begin{figure}[!ht]
 \caption{How Hahner Twins Rank in Difference-in-Differences}
 \label{fig:diffdiff}
 \begin{subfigure}{.5\textwidth}
 \includegraphics[width=\textwidth, keepaspectratio]{diff_in_diff_1.pdf}
 \end{subfigure}
 \begin{subfigure}{.5\textwidth}
 \includegraphics[width=\textwidth, keepaspectratio]{diff_in_diff_2.pdf}
 \end{subfigure}
\end{figure}




\newpage
\subsection*{Simulating the Marathon}

The Hahner sisters claimed that they ran the marathon independent of each other.  Their simultaneous finish was no more than a product of ability and luck.  One might expect such a finish given that their personal best marathons are within 114 seconds of each other.  Add a little bit of luck to the occasion and, as the twins have claimed all along, one might easily argue that the simultaneous finish is simply a joyous coincidence rather than a wholly implausible event.

This begs the question: if we take into consideration their similarities as runners as well as some natural variation in finishing a marathon, what are the odds that the twins unintentionally finished at the same time?   To answer this question we would want to know the counterfactual distribution of potential finishing times that would have occurred had the twins independently and repeatedly run the marathon over again.  Holding the twins' similarities constant, this would establish the set of potential outcomes that would have occurred as result of natural variation in the finishing the race.   Then we could use this distribution of outcomes to determine the likelihood that a simultaneous finish (or at least a near-simultaneous finish) occurred by chance alone.   If the twins rarely finished the race together in this counterfactual, then one should be skeptical that such a finish occurred without coordination.

Unfortunately, it is not possible to rerun the women's marathon in Rio to establish this counterfactual distribution.  However, we can attempt to simulate it, by estimating the distribution of every other runner's results -  conditioned on their prior abilities - and drawing from that distribution in order to observe the likelihood that Lisa and Anna finish simultaneously.  

To estimate the conditional distribution of every runner's results, we must assume that each runner's time is distributed with a mean that is a quadratic function of $X_i$  and $\beta_j$, such that

$$Y_i \mid X_i \sim N(\beta_0 + \beta_1X_{i} + \beta_2X^2_{i}  + e_i, \sigma).$$

This means that we are assuming that each runner's time is drawn from the normal distribution with a mean that is a quadratic function of their personal best and variance that is constant across all runners.  We simply estimate $\beta_j$ and $\sigma$ using ordinary least squares.  We then draw every runner's time from this estimated distribution, conditioned on their personal best outcome.  Once a race is simulated, we locate Anna and Lisa and record the time between their finishes and the difference in their rank.  We then simulate a new race - drawing a new set of finishing times - and record the same quantities.  After repeating this procedure 10,000 times, we plot the distribution in Figure \ref{fig:simdiff}.  The steps of the simulation are detailed below.


\begin{enumerate}
\item Ignoring twins and triplets, estimate with least squares a
  linear model that predicts a runner's finishing time ($Y_i$) based
  on her pre-Olympic personal best time ($X_i$) and her best time
  squared.
\item Extract coefficient vector and the covariance matrix from this
  model.
\item For each simulated rate, draw intercept and slope estimates
  $\beta_0$, $\beta_1$, and $\beta_2$, respectively, from a
  multivariate normal distribution with mean equal to the estimated
  coefficient vector and covariance equal to the estimated covariance
  matrix.
\item For each runner, draw an error ($e$) from a normal distribution
  with mean zero and a standard deviation equal to the standard
  deviation of the model's residuals ($\sigma$).
\item For each runner, predict the final result by combining the randomly
  generated beta coefficients and error terms, $\hat{y} = \beta_0 +
  \beta_1\,X_i + \beta_2\,X_i^2 + e$.
\item Eliminate each runner from the race with a probability equal to
  the fraction of runners who did not finish the marathon.
\item Calculate the difference in time and difference in ranking
  between Anna Hahner and Lisa Hahner, assuming both finished the
  race.
\item Repeat above steps ten thousand times.
\item Plot a histogram of the results in Figure \ref{fig:simdiff}.
\end{enumerate}



\begin{figure}[!ht]
  \centering
  \caption{Distribution of simulated results for the Hahner twins}
  \label{fig:simdiff}
  \begin{subfigure}{.45\textwidth}
    \includegraphics[width=\textwidth,
    keepaspectratio]{simulated_time.pdf}
    \caption{Simulated differences in finishing times}
    \label{fig:simulatedfinishtimes}
  \end{subfigure}
  \begin{subfigure}{.45\textwidth}
    \includegraphics[width=\textwidth, keepaspectratio]{simulated_rank.pdf}
    \caption{Simulated differences in places}
    \label{fig:simulatedranks}
  \end{subfigure}
\end{figure}


A histogram of the differences between Anna and Lisa's simulated finishes can be found in Figure \ref{fig:simdiff}.  The first plot shows the distribution of the absolute differences in time.  Differences are grouped in 30 second bins along the x-axis and the count for each bin is denoted by the vertical length of bar along the y-axis.  The second plot shows the distribution of the absolute differences in ranking.  Differences are grouped as single units ranging from no runner between the twins to nearly 120 runners between them. 

The results plotted in the histograms question the credibility of Anna and Lisa's story.  The results suggest that  a simultaneous finish would be very rare if Anna and Lisa had run independently of each other.  For example, in less than 300 of the 10,000 simulated races did Anna and Lisa finished within 30 seconds of each other .  And they finished in consecutive rank in less than 175 of the 10,000 races.  Therefore, the close finish that we observed, where Anna and Lisa crossed the finish line one-after-the-other, would be very unlikely if we were to believe the twins raced independently of each other.  

The results holds even if you replace their personal best time with their time halfway through the race.  While this reduces the variation in the predicted outcome of each runner and, therefore, reduces the expected distance between Anna and Lisa, it is still quite rare for Anna and Lisa to finish simultaneously or consecutively.




\begin{figure}[!ht]
  \centering
  \caption{Distribution of simulated results for the Hahner twins (using halfway split as predictor)}
  \label{fig:simdiffhalf} 
  \begin{subfigure}{.45\textwidth}
    \includegraphics[width=\textwidth,
    keepaspectratio]{simulated_time_half.pdf}
    \caption{Simulated differences in finishing times}
    \label{fig:simulatedfinishtimes}
  \end{subfigure}
  \begin{subfigure}{.45\textwidth}
    \includegraphics[width=\textwidth, keepaspectratio]{simulated_rank_half.pdf}
    \caption{Simulated differences in places}
    \label{fig:simulatedranks}
  \end{subfigure}
\end{figure}




\section*{Conclusion}


\newpage
\section*{Further reading}

\begin{itemize}

\item \url{http://www.telegraph.co.uk/olympics/2016/08/17/german-twins-criticised-for-finishing-olympic-marathon-fun-run-h}

\item
  \url{https://www.nytimes.com/2016/08/17/sports/olympics/twins-finish-marathon-hand-in-hand-but-their-country-says-they-crossed-a-line.html}

\item
  \url{https://www.welt.de/sport/olympia/article157669264/Das-falsche-Laecheln-der-deutschen-Lauf-Zwillinge.html}
\end{itemize}

\end{document}

