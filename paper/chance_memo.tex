\documentclass[12pt]{article}
\sloppy
\usepackage{times}
\usepackage{url}
\usepackage{arydshln}

%% uncomment to use a watermark
%\usepackage{draftwatermark}
%\SetWatermarkScale{1}


\begin{document}
\thispagestyle{empty}


\noindent {\Huge Memo}\\[0.5in]

\noindent
\begin{tabular}{ll}
  \textbf{To:} &  Scott Evans, editor, \emph{CHANCE}\\
  \textbf{From:} & David Cottrell and Michael C.\ Herron, Dartmouth College\\
  \textbf{Date:} & \today \\
  \textbf{Re:} &  Revisions to women's marathon submission
\end{tabular}

\vspace{.1in}
\hrule
\vspace{.2in}

\bigskip

This memorandum is a response to your February 16, 2017, email
regarding our manuscript, ``All in the family: German twin finishing
times in the 2016 women's Olympic marathon.'' Thank you very much for
inviting us to submit a revision of our manuscript to \emph{CHANCE},
and this memorandum explains how we have updated it in light of
comments from you and the Editorial Board.

\begin{itemize}
\item ``Am not convinced of quadratic model. Two points in upper right
  may have influenced this. What is p-value for linear vs quadratic?
  What happens if those 2 points are removed? Could also simply plot
  differences from personal best as frequency dist'n. Could also plot
  projected from half vs actual.''

\item ``I printed this out in black and white. Would help see them if
  Hahners were darkest color.''

\item ``a similar conclusion is reached?''

  We changed the language in the paper so it uses this phrasing.

\item ``pairs are not completely independent as same runner can be in
  different pairs. Not as though runners are drawn with replacement.''

\item ``due to subsequent DNF's?''

  We changed the language in the paper so it uses this phrasing.

\item ``I don't get this part. If Yi is function of Xi, personal best,
  won't estimates be same for each simulation?  What I would try for
  an analysis would be look at distribution of finishing times,
  expecting it to be normal, as most human performance is fairly
  normal (most are average, some above, some below). I'd use the
  observed mean and variance of the finishers plus a random error, and
  see what this theoretical dist'n looks like. How often are finishers
  within x seconds of each other? Could also use historical marathon
  data from other races.''

\item There are errors in figure labels, ``5a'' and ``5b''

  We apologize for this oversight and have fixed the associated
  labeling errors.

\item ``Would help to expand left side bins, as these are the finishes
  of main interest.''

\item ``what are interval widths in seconds? Hard to see.''

\item ``Suddenly Latin?''

  We have changed the Latin to, all things equal.

\item ``likely''


\end{itemize}

 \end{document}
