\documentclass[12pt]{article}
\sloppy
\usepackage{times}
\usepackage{xr-hyper}
\usepackage{url}
\usepackage{arydshln}

\externaldocument{paper}

%% uncomment to use a watermark
%\usepackage{draftwatermark}
%\SetWatermarkScale{1}


\begin{document}
\thispagestyle{empty}


\noindent {\Huge Memo}\\[0.5in]

\noindent
\begin{tabular}{ll}
  \textbf{To:} &  Scott Evans, editor, \emph{CHANCE}\\
  \textbf{From:} & David Cottrell and Michael C.\ Herron, Dartmouth College\\
  \textbf{Date:} & \today \\
  \textbf{Re:} &  Revisions to women's marathon submission
\end{tabular}

\vspace{.1in}
\hrule
\vspace{.2in}

\bigskip

This memorandum is a response to your February 16, 2017, email
regarding our manuscript, ``All in the family: German twin finishing
times in the 2016 women's Olympic marathon.'' Thank you very much for
inviting us to submit a revision of our manuscript to \emph{CHANCE},
and this memorandum explains how we have updated it in light of
comments from you and the Editorial Board.

\begin{itemize}
\item ``Am not convinced of quadratic model. Two points in upper right
  may have influenced this. What is p-value for linear vs quadratic?
  What happens if those 2 points are removed? Could also simply plot
  differences from personal best as frequency dist'n. Could also plot
  projected from half vs actual.''

  The $t$-value for the quadratic term is -1.88 and the corresponding
  $p$-value is 0.062, which is not particularly compelling.  We have
  accordingly dropped the quadratic terms in the scatter plots on p.\
  4 and, implicitly, from the Studentized residual plots on p.\ 6.  In
  terms of the latter suggestion about, Figure 1b contains a plot of
  half marathon split times versus actual finishing times.
  %% Note: xr labels are
  %% breaking; looks like some sort of
  %% label clash with another package.
  %% Hardcoding here is poor style.
  
  % > fit <- lm (FINAL ~ PB + I(PB^2), data = df)
  % > print (summary(fit))
  
  % Call:
  % lm(formula = FINAL ~ PB + I(PB^2), data = df)
  
  % Residuals:
  % Min      1Q  Median      3Q     Max 
  % -817.64 -288.70  -92.13  224.05 1319.33 
  
  % Coefficients:
  % Estimate Std. Error t value Pr(>|t|)   
  % (Intercept) -1.257e+04  6.417e+03  -1.958  0.05233 . 
  % PB           3.633e+00  1.336e+00   2.719  0.00744 **
  % I(PB^2)     -1.309e-04  6.946e-05  -1.884  0.06181 . 
  % ---
  % Signif. codes:  0 ‘***’ 0.001 ‘**’ 0.01 ‘*’ 0.05 ‘.’ 0.1 ‘ ’ 1
  
  % Residual standard error: 417.5 on 130 degrees of freedom
  % (24 observations deleted due to missingness)
  % Multiple R-squared:  0.6538,	Adjusted R-squared:  0.6484 
  % F-statistic: 122.7 on 2 and 130 DF,  p-value: < 2.2e-16
  
\item ``Marathon finish may be linearly related to the fastest
  previous marathon with perhaps age in a model. An interesting though
  experiment may be, if they were blindfolded, would they have
  finished together?''

  We have added age to our simulation model, and the results are
  qualitatively equivalent.  In addition, if one adds age to the
  linear specification depicted in our basic scatter plot (now without
  a quadratic term, as noted above), age is a significant predictor of
  marathon finishing time, even controlling for personal best
  ($t \approx 2.426$, $p \approx 0.017$).  Older runners are slower,
  all things equal.

% Call:
% lm(formula = FINAL ~ PB + years2marathon, data = df)

% Residuals:
%     Min      1Q  Median      3Q     Max 
% -779.88 -281.31  -94.71  222.21 1319.29 

% Coefficients:
%                  Estimate Std. Error t value Pr(>|t|)    
% (Intercept)    -1.366e+03  7.430e+02  -1.838   0.0684 .  
% PB              1.147e+00  7.222e-02  15.876   <2e-16 ***
% years2marathon  1.799e+01  7.413e+00   2.426   0.0166 *  
% ---
% Signif. codes:  0 ‘***’ 0.001 ‘**’ 0.01 ‘*’ 0.05 ‘.’ 0.1 ‘ ’ 1

% Residual standard error: 413.9 on 130 degrees of freedom
%   (23 observations deleted due to missingness)
% Multiple R-squared:  0.6597,	Adjusted R-squared:  0.6545 
% F-statistic:   126 on 2 and 130 DF,  p-value: < 2.2e-16


\item ``I printed this out in black and white. Would help see them if
  Hahners were darkest color.''

  We have changed the color scheme for the dots in our figures.  The
  German Hahner twins are now shown in black, the North Korean Kim
  twins in red, and the Estonian Luik triplets in blue.  Each color is
  present in its country's respective flag.
  
\item ``a similar conclusion is reached?''
  
  We changed the language in the paper so that it uses this phrasing.
  
\item ``pairs are not completely independent as same runner can be in
  different pairs. Not as though runners are drawn with replacement.''

  We agree with the thrust of this comment, and our manuscript now
  emphasizes the lack of independence.  Insofar as dependence in
  sampling typically diminishes variance, our lack of independence in
  the pair plots should make our results conservative. This means that
  that the observed Hahner and Kim compression in final results,
  conditional on prior best marathon times, is even more unusual.

\item ``due to subsequent DNF's?''

  We changed the language in the paper so that it uses this phrasing.
  
\item ``I don't get this part. If Yi is function of Xi, personal best,
  won't estimates be same for each simulation?  What I would try for
  an analysis would be look at distribution of finishing times,
  expecting it to be normal, as most human performance is fairly
  normal (most are average, some above, some below). I'd use the
  observed mean and variance of the finishers plus a random error, and
  see what this theoretical dist'n looks like. How often are finishers
  within x seconds of each other? Could also use historical marathon
  data from other races.''

  We fully agree with this comment, and indeed the intuition offered
  by the referee was the motivation behind our simulation.  To make
  this clear, we added a paragraph after the simulation steps, and
  this paragraph explains why finishing times (and order) will vary
  both within and across simulations.  Then, the plots we generated
  (see Figures 5 and 6) describe how often the Hahner twins finish
  in close proximity of each other.

  We have thought about historical marathon data but are hesitant to
  go down that route.  The set of runners who competed in Rio is
  unique to this race, and marathon finishes are confounded by course
  and local conditions.
  
\item There are errors in figure labels ``5a'' and ``5b''
  
  We apologize for this oversight and have fixed the associated
  labeling errors.

\item ``Would help to expand left side bins, as these are the finishes
  of main interest.''
  
  We agree with this suggestion.  Accordingly, in the histograms that
  describe simulation results, we have highlighted left side bins in
  red and annotated the plots with descriptions of the sizes of key
  areas.  This emphasizes the finishes that are most important for our
  results.

\item ``what are interval widths in seconds? Hard to see.''

  We agree that this is hard to see.  The text now points out that bin
  width is 30 seconds.

\item ``Suddenly Latin?''
  
  We have changed the Latin text to, all things equal.
  
\item ``likely''

  We apologize for the confusion here, but we are not sure what the
  referee is requesting.  In the marked-up PDF document, the word
  ``likely'' appears over the following sentence: ``Their finish—one
  second between the two women—was an extremely low probability
  event.''  We are not sure how to incorporate the referee's request
  but are happy to do so if given more information.

\end{itemize}

 \end{document}
